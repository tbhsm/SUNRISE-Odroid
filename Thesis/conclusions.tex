\section{Conclusions}\label{sec:conclusions}
\subsection{General conclusion}
The goal of the overall project was to design software packages and an intuitive user-interface for an e-bike charging station. Users can make a reservation in the user-interface after which they can charge their e-bike in the charging station. The main focus in the design process described in the sections above was the low-level design of data acquisition and control of the chargers using an ODROID minicomputer.\\

All targets with respect to the data acquisition were reached: the C-program on the ODROID is able to read all data generated by the power electronics, weather station and temperature sensors. Using the GPIO pins from the ODROID the relais can be controlled, which turn on/off the chargers. This concept was succesfully tested using LEDs since the chargers were not finished yet. Furthermore, a charger control Finite State Machine was designed and tested. This controller has buffers so the charger state does not immediately change when a cable is (dis)connected (for example, plugging the cable in badly could cause it to turn on and off immediately), as well as a timer between logging in on the server and plugging in. At this moment it is not yet possible to control the local display which will be mounted in the charging station, because there is no webpage yet to control. A conceptual design for the control mechanism is available though, including the working WebGL on the ODROID for the graphical elements of the webpage.\\

During tests of multiple days the C-code was succesful with respect to robustness and memory leaks. Moreover, additional Bash scripts were written to support the functionality of the code. For example, a script to solve memory leaks caused by the remote desktop was written, as well as a script to immediately start certain programs on the ODROID after a reboot (a full-screen browser and the C-code). Finally, an installation and configuration manual was made. Using this manual it is possible set up a new ODROID using some easy steps. After performing these steps the C-code should be able to run on the ODROID.\\
