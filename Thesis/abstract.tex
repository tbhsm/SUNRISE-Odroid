\begin{abstract}
%\setlength\parindent{0pt}
In the summer of 2016, a solar-powered charging station for e-bikes (both wired and wireless) and e-scooters will be built on the campus of Delft University of Technology. This station is also meant to provide a testing facility for future projects. The station has a number of systems that generate data: the power electronics (including the solar panels, batteries and grid-connection), a weather station, solar panel temperature sensors and the connected vehicles.\\

The SUNRISE (Smart Unified Networking Rig for an Integrated Solar E-bike charger) project is a BSc final project that should log and display all this data through an administrator page, a website and a local display. It should also be able to manage e-bike users that would like to connect to the station.\\

Team ODROID was responsible for the local data management and charger control. The team managed to fetch all necessary data from the power electronics, temperature probes and weather station with C-code running on an ODROID C1+. Furthermore, the developed system was designed from the start to be robust (since it will be running 24/7) and has been thoroughly tested for possible malfunctions. The system has also been made resistant against power and internet outages. Moreover, the system can register e-bike cable connections, which it uses to fully control the chargers. It also provides data necessary to display system information on a local display. Finally, the entire system is remotely controllable through a remote desktop.
\end{abstract}
